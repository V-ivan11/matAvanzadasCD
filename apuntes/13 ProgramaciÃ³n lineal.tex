\PassOptionsToPackage{unicode=true}{hyperref} % options for packages loaded elsewhere
\PassOptionsToPackage{hyphens}{url}
%
\documentclass[]{article}
\usepackage{lmodern}
\usepackage{amssymb,amsmath}
\usepackage{ifxetex,ifluatex}
\usepackage{fixltx2e} % provides \textsubscript
\ifnum 0\ifxetex 1\fi\ifluatex 1\fi=0 % if pdftex
  \usepackage[T1]{fontenc}
  \usepackage[utf8]{inputenc}
  \usepackage{textcomp} % provides euro and other symbols
\else % if luatex or xelatex
  \usepackage{unicode-math}
  \defaultfontfeatures{Ligatures=TeX,Scale=MatchLowercase}
\fi
% use upquote if available, for straight quotes in verbatim environments
\IfFileExists{upquote.sty}{\usepackage{upquote}}{}
% use microtype if available
\IfFileExists{microtype.sty}{%
\usepackage[]{microtype}
\UseMicrotypeSet[protrusion]{basicmath} % disable protrusion for tt fonts
}{}
\IfFileExists{parskip.sty}{%
\usepackage{parskip}
}{% else
\setlength{\parindent}{0pt}
\setlength{\parskip}{6pt plus 2pt minus 1pt}
}
\usepackage{hyperref}
\hypersetup{
            pdfborder={0 0 0},
            breaklinks=true}
\urlstyle{same}  % don't use monospace font for urls
\usepackage{longtable,booktabs}
% Fix footnotes in tables (requires footnote package)
\IfFileExists{footnote.sty}{\usepackage{footnote}\makesavenoteenv{longtable}}{}
\setlength{\emergencystretch}{3em}  % prevent overfull lines
\providecommand{\tightlist}{%
  \setlength{\itemsep}{0pt}\setlength{\parskip}{0pt}}
\setcounter{secnumdepth}{0}
% Redefines (sub)paragraphs to behave more like sections
\ifx\paragraph\undefined\else
\let\oldparagraph\paragraph
\renewcommand{\paragraph}[1]{\oldparagraph{#1}\mbox{}}
\fi
\ifx\subparagraph\undefined\else
\let\oldsubparagraph\subparagraph
\renewcommand{\subparagraph}[1]{\oldsubparagraph{#1}\mbox{}}
\fi

% set default figure placement to htbp
\makeatletter
\def\fps@figure{htbp}
\makeatother


\date{}

\begin{document}

\hypertarget{programaciuxf3n-lineal}{%
\section{Programación lineal}\label{programaciuxf3n-lineal}}

En 1949 George B. Danterg publicó el método simplex para resolver
programas lineales, como de sus investigaciones en transporte óptimo.

El método simplex de programación lineal posee una amplia aceptación
debido a 2 razones principalmente.

Su habilidad para \textbf{modelar} problemas complejos (núcleos
variables)

Su capacidad para \textbf{producir} soluciones en una cantidad de tiempo
razonable.

En esta parte de nuestro curso introduciremos el concepto de "Problema
de programación lineal"

\hypertarget{asumpciones-y-ejemplos}{%
\subsection{Asumpciones y ejemplos}\label{asumpciones-y-ejemplos}}

\hypertarget{definiciuxf3n-1}{%
\subsubsection{Definición}\label{definiciuxf3n-1}}

\begin{quote}
Un problema de programación lineal es un problema de minimización o
maximización de una función lineal con restricciones (ecuaciones e
inecuaciones lneales), es decir

minimizar/maximizar \(\qquad ax_1+ax_2+\cdots+a_nx_n\)

Sujeto a

\[ax_1+ax_2+\cdots+a_nx_n \ge b_1\\

ax_1+ax_2+\cdots+a_nx_n \ge b_2\\
\vdots\\

a_{m1}x_1+a_{m1}x_2+\cdots+a_{mn}x_n \ge b_m \\
NN \qquad x_1, x_2,\cdots,x_n\ge0\]

\mbox{}%
\hypertarget{notaciuxf3n}{%
\paragraph{Notación}\label{notaciuxf3n}}

\[Z = 
\begin{bmatrix}
 c_1 \\
 c_2 \\
 \vdots \\
 c_n
\end{bmatrix}, 
\quad X = 
\begin{bmatrix}
 x_1 \\
 x_2 \\
 \vdots \\
 x_n
\end{bmatrix}, \rightarrow \text{Variabes de desición}\]

\[\therefore Z^tX = c_1x_1+c_2x_2+\cdots+c_nx_n \rightarrow \text(Coeficientes de costo)\]

\[\sum_{j=1}^n a_{ij}x_1 \ge b_i \quad \rightarrow \quad \text{iésima restricción}\\
a_{ij} \quad i = 1,2,\cdots,m \qquad j=1,2,\cdots,n\\
A = \left(a_{ij}\right)^{nk} \quad \rightarrow \quad \text{Matriz de restricción}\\

\text{sujeto a }\quad A\mathbb x \ge\mathbb b \quad (\iff(A_x)_i \ge b_i)\\
\qquad \mathbb x \ge 0 \quad (\iff x_i \ge 0)\]
\end{quote}

\hypertarget{definiciuxf3n-2}{%
\subsubsection{Definición}\label{definiciuxf3n-2}}

\begin{quote}
Un conjunto de variables \(x_1, x_2, \cdots, x_n\) que satisface las
restricciones \(1,2,\cdots,m\) así como la restricción \(NN\) se llama
parte factible o vector factible.
\end{quote}

El conjunto de todos los puntos anteriores se denomina región.

\hypertarget{ejemplo-1}{%
\paragraph{Ejemplo: }\label{ejemplo-1}}

Minimizar \(2x_1+5x_2\), sujeto a
\(x_1+x_2\ge 6 \qquad -x_1-2x_2 \ge -18 \qquad x_1,x_2 \ge 0\).

Con el fin de representar un problema de optimización como uno de
programación lineal debemos asumir algunas cosas que están implícitas en
la formulación previamente discutidas.

\hypertarget{1-proporcionalidad}{%
\subsubsection{1) Proporcionalidad}\label{1-proporcionalidad}}

Dada una variable \(x_j\), su contribución con el costo, \(C_jx_j\) y su
contribución a la \(iésima\) restrición es \(a_{ij}x_j\).

Esto significa que la variación \(x_j\) es proporcional a la variación
de contribución al costo y a la variación .

\[x_j r (jx_j\cap a_ij)x_j\]

\hypertarget{2-aditividad}{%
\subsubsection{2) Aditividad}\label{2-aditividad}}

Esta asumpción garantiza que el costo total es la suma de los costos
individuales y que la contribución total a la \(iésima\) restricción es
la suma de las contribuciones individuales.

\hypertarget{3-divisibilidad}{%
\subsubsection{3) Divisibilidad}\label{3-divisibilidad}}

Las variables de decisión pueden ser divididas en niveles continuos.

\hypertarget{manipulaciuxf3n-algebraica}{%
\subsection{Manipulación algebraica}\label{manipulaciuxf3n-algebraica}}

El uso de operaciones algebraicas son muy útiles para llevar un problema
de una forma a otra y así poder estudiarlos a detalle. Veamos algunos
ejemplos que podemos considerar.

\hypertarget{desigualdades-y-ecuaciones}{%
\subsubsection{Desigualdades y
ecuaciones}\label{desigualdades-y-ecuaciones}}

Una desigualdad puede ser fácilmente transformada en una ecuación. Para
ilustrarlo consideremos la notación.

\[\sum_{j=1}^na_{ij}x_j \ge b_i\qquad (1)\]

Esta restricción puede ser transformada en una ecuación de

\[x_{n+1}=\sum_{j=1}^n a_{ij}x_j-b_i\ge0\]

Luego \((1)\) es equivalente a:

\[\sum_{i=1}^n a_{ij}x_j-x_{n+1}=b_i\]

similarmente la restricción

\[\sum_{i=1}^n a_{ij}x_j+x_{n+1}=b_i \text{con}\\
x_{n+1}= b_i - \sum_{i=1}^n a_{ij}x_j\]

\hypertarget{no-negatividad-de-las-variables}{%
\subsubsection{No negatividad de las
variables}\label{no-negatividad-de-las-variables}}

En los problemas prácticos las variables usualmente representan
cantidades físicas y por lo tanto deben ser no negativos. Si una
variable \(x_j\) no tiene restricciones entonces esta puede ser
reemplazada por \(x_j = x_j.x_j''\).

donde \(x_j', x_j''\ge 0\) y por definición

\[x_j' = \left\lbrace
\begin{array}{ll}
x_j \qquad x_j \ge 0\\
0 \qquad x_j \le 0 
\end{array}
\right. \\\\
x_j'' = \left\lbrace
\begin{array}{ll}
-x_j \qquad -x_j \ge 0\\
0 \qquad -x_j \le 0 
\end{array}
\right. \\\]

Si la restricción del tipo

\[x_j \ge l_j\]

entonces:

\begin{eqnarray*}
x_j'&=& x_j-l_j \ge 0\\
&si& x_j\le u_j
\end{eqnarray*}

donde \(u_j\ge 0\) entonces

\[x_j=u_j-x_j\]

Num 2x, sujeto a \(Ax\ge b\)

\[NN \qquad x_1.x_2,\cdots,x_j \ge 0\]

\hypertarget{problemas-de-minimizaciuxf3n-y-maximizaciuxf3n}{%
\subsubsection{Problemas de minimización y
maximización}\label{problemas-de-minimizaciuxf3n-y-maximizaciuxf3n}}

\[\text{máximo} \sum_{j=1}^n c_jx_j = -\text{mínimo} \sum_{j=1}^n -c_jx_j\\
(c_1, c_2, \cdots, c_n) = z\]

\begin{longtable}[]{@{}lll@{}}
\toprule
& Problema de Minimización & Problema de Maximización\tabularnewline
\midrule
\endhead
Forma Estándar &
\(\text{sujeto a j} \\ \sum_{j=1}^n a_jx_j=b_i\\ x_j \ge 0\\ (i,j) \in \mathbb N⁺\)
&
\(\text{sujeto a j} \\ \sum_{j=1}^n a_jx_j=b_i\\ x_j \ge 0\\ (i,j) \in \mathbb N⁺\)\tabularnewline
Forma Canónica &
\(\text{sujeto a j} \\ \sum_{j=1}^n a_jx_j\ge b_i\\ x_j \ge 0\\ (i,j) \in \mathbb N⁺\)
&
\(\text{sujeto a j} \\ \sum_{j=1}^n a_jx_j\le b_i\\ x_j \ge 0\\(i,j) \in \mathbb N⁺\)\tabularnewline
\bottomrule
\end{longtable}

\hypertarget{ejemplo-1}{%
\paragraph{Ejemplo 1}\label{ejemplo-1}}

Un molino agricultor manufactura comida para que esto se hace mezclando
varios ingredientes, tales como el maíz, cal, alfalfa. El mezclado se
lleva a cabo de tal forma que el alimento alcance niveles dados para
diferentes tipos de nutrientes como, proteínas, carbohidratos y
vitamínas.

De manera más concreta, supongamos \(n\) ingredientes
\(j = 1,2,\cdots,n\) y \(m\) nutrientes. Sea el costo unitario por
ingrediente \(j\), el valor \(c_j\) y sea la cantidad \(x_j\) de
ingrediente \(j\) para ser usado. Por tanto el costo total es:

\[\sum_{j = 1}^n c_jx_j\]

Si la cantidad de producto final es \(b\), entonces debemos tener:

\[\sum_{j=1}^n x_j= b\]

Además suponga que \(a_{ij}\) es la cantidad de nutrientes \(i\)
presente en el ingrediente \(j-ésimo\) y que los límites superior e
inferior del nutriente \(i\) en una unidad de comida son \(l_i'\) y
\(u_i'\) respecticamente

\[l_i'b\le \sum_{j = 1}^n a_{ij}x_j \le a_i'b\]

Finalmente, debido a la escazes, supongamos que el molino no puede
adquirir más de \(u_j\) unidades del ingrediente \(j\).

\[\text{minimizar} \sum_{j=1}^n c_jx_j\\
\text{con restricciones:}\\
l_i'b\le \sum_{j = 1}^n a_{ij}x_j \le a_i'b\\
0\le x_j \le u_j \qquad (i,j)\in \mathbb N⁺\\
x_1+x_2+\cdots+x_n=b\]

Acortando

\[\text{minimizar } cx\\
\text{sujeto a } ba \ge A_x \ge bl\\
u-x\ge0\\
xl = b\]

\hypertarget{ejemplo-2-el-problema-del-transporte}{%
\paragraph{Ejemplo 2: El problema del
transporte}\label{ejemplo-2-el-problema-del-transporte}}

Cantidades \(a_1, a_2, \cdots, a_m\) respectivamente de un cierto
producto que serán enviadas de cada uno de \(m\) puertos y se recibirán
en \(n\) destinos con la distribución \(b_1,b_2,\cdots, b_n\) . Digamos
que una compañía de café produce café en \(m\) plantas. El cafe es
mandado todas las semanas a \(n\) tiendas para su distribución.
Supongamos que el costo por kilogramo de café al cual la planta
\(i-ésima\) vende a la tienda \(j-ésima\) es \(c_{ij}\). Además suponga
que la capacidad de producción de la planta \(i-ésima\) es simplemente
\(a_i\) y la demanda de la tienda \(j-ésima\) es \(b_j\).

Se desea encontrar la distribución de producción \((x_{ij})_{ij}^{mn}\)
que minimiza el costo de envío .

\[\text{Minimizar} \sum_{i=1}^m \sum_{j=1}^n c_{ij}x_{ij}\\
\text{sujeto a } \sum_{j=1}^n x_{ij} \le a_i, \qquad j = 1,2,\cdots,n\\
\sum_{i=1}^m x_{ij} = b_j \qquad i = 1,2,\cdots,m\\
x_{ij}\ge0\]

\hypertarget{definiciuxf3n-3}{%
\subsubsection{Definición}\label{definiciuxf3n-3}}

\begin{quote}
Un conjunto \(S \subset \mathbb R^n\), se dice convexo si
\(\forall x,y\in S\).

\[\lambda_x + (1-\lambda)y\in S, \forall \lambda\in[0,1]\]
\end{quote}

\hypertarget{ejemplo-2}{%
\paragraph{Ejemplo}\label{ejemplo-2}}

Los siguientes conjuntos son ejemplos de conjuntos convexos

\begin{itemize}
\item
  \textbf{a)} \(\{(x_1,x_2) | x_1²+x_2² \in 1\}\)
\item
  \textbf{b)} \(\{x | Ax = b\}\) donde \(A\) es una muestra de dimensión
  \(m\times n\) y \(b\) un vector de dimensión \(m\)
\item
  \textbf{c)} \(\{x | Ax = b, x \ge 0 \}\), donde A es una matriz de
  dimensión \(m\times n\) y \(b\) es un vector de dimensión \(m\).
\end{itemize}

\[\mathbb X_1, \mathbb X_2 \quad [\lambda \mathbb x_1 + (1-\lambda) \mathbb x_2]\]

\textbf{Nota:}

La intersección de cualquier familia de conjuntos convexos es convexo.

\hypertarget{definiciuxf3n-4}{%
\subsubsection{Definición}\label{definiciuxf3n-4}}

\begin{quote}
Un punto \(x\) en un conjunto convexo \(X\) es llamado un punto extremo
de \(X\), si \(x\) no puede ser representado como una combinación lineal
convexa estricta de puntos distintos de \(x\).
\end{quote}

\hypertarget{hiperplano-y-semiplano}{%
\subsubsection{Hiperplano y Semiplano}\label{hiperplano-y-semiplano}}

Un hierplano en \(\mathbb R^n(E^n)\) generaliza la noción de línea en
\(\mathbb R²\) y la noción de un plano en \(\mathbb R³\) . Un hiperplano
\(H\) en \(\mathbb R^n\) es un conjunto de \(h\) convexo.

\[H=\{x|px = k\}\]

Si \(x_0 \in H\), entonces \(Px_0=k\), y todo \(x\in H\) cumple que
\(px=k \quad \therefore\)

\[P(x-x_0)= 0\]

Luego \(H = \{x | P(x-x_0 = 0, x_0 \in H)\}\)

\textbf{Nota:} Cualquier biplano es un conjunto convexo y donde a
\(\mathit E^n\) en 2 regiones llamadas semiplanos, a sobre,

\[\{x | P(x-x_0 \ge 0, x_0 \} = \{x| px \ge k\}\\
\{x | P(x-x_0 \le 0, x_0 \} = \{x| px \le k\}\]

Otro ejemplo de conjunto convexo es un rayo

\[\{x_0 + \lambda d | \lambda \ge 0\}\]

Dado un conjunto convexo, un vector no cero es llamado dirección del
conjunto si para cada \(x_0\), el rayo \(\{x_o + \lambda d | x \ge 0\}\)

\[X=\{x | Ax = b, x\ge 0\}\]

Se tiene que \(d\) es dirección de \(x\) si y sólo si, las componentes

\[d\ge0,\qquad d\not=0, \qquad Ad=0\]

Y que dicho conjunto de direcciones es también convexo. \emph{Nota: la
demostración de este hecho se encuentra definida en nuestro libro de
texto.}

Una dirección extrema de un conjunto convexo es una dirección del
conjunto que no puede ser expresada como combinación lineal positiva de
2 direcciones del conjunto.

\end{document}
